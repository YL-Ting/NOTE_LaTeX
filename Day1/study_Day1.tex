% Preamble - to set up the global setting of the file

% to set type of this file ex:beamer(like PPT), article(paper)...
% can use optional [] to default the font size,10pt,11pt,12pt is most use
\documentclass[10pt]{article}

% what package you use,the [] is optional setting
\usepackage[margin=1.25in]{geometry}
\usepackage{amsmath, amssymb}
\usepackage{graphicx,ulem}

% to set the length of tab before a paragraph
\setlength{\parindent}{1cm}
\author{YL TING}
\date{\today}
\title{Learning LaTex - Day1}

% Document itself
\begin{document}
    \maketitle
    % insert photo
    \begin{center}
        \includegraphics[width=0.3\linewidth]{test_photo.png}
    \end{center}

    \newpage
    
    % font size
    \begin{center}
        {\LARGE Font Size}\\[0.25cm]
        {\tiny tiny}\\
        {\scriptsize scriptsize}\\
        {\footnotesize footnotesize}\\
        {\normalsize normalsize}\\
        {\large large}\\
        {\LARGE LARGE}\\
        {\huge huge}\\
        {\Huge Huge}\\[1cm]
    \end{center}

    \begin{center}
        {\LARGE Font Style}\\[0.25cm]
        Normal text\\[0.2cm]
        \textbf{Bold text} \\[0.2cm]
        \textit{Italic text}\\[0.2cm]
        \underline{Underlined text}\\[0.2cm]
        \emph{emphasis text}\\[0.2cm]
        \emph{\emph{double emphasis text}}\\[0.2cm]
        % below should use ulem package
        \uline{underlined text but will not over the block margin,useful for long underlined string}\\
        \uuline{Double underlined text}\\[0.2cm]
        \uwave{Wavy underlined text}\newline %\newline can not use []
    \end{center}

    % math mode
    % Display Style Math - Puts math on center of display
    \begin{center}
        {\LARGE Display Style Math}\\
    \end{center}
    %   type 1
    \[f(x) = (x+2)^2 - 9\]
    %   type 2-1
    %       - the & is use to align the math expression
    %       - the * back of align can cancel the number of fuction
    %         or use \nonumber (only use on the end of line)
    \begin{align}
        f(x) & = a_2 x^{ky} + a_1 x +a_0 \nonumber \\
             & = x^2 + 4x - 5
    \end{align}
    %   type 2-2 
    \begin{align*}
        2x+1 & = 9 & 3y-2 & = -5 & -5z-8 & = 3 \\
        2x & = 8 & 3y & = -3 & -5z & = -5 \\
        x &= 4 & y &= -1 & z &= -1
    \end{align*}

    % Inline or Text Style Math - math stays in line
    \begin{center}
        {\LARGE Inline or Text Style Math}\\
    \end{center}
    %   type 1 - enclose to pair of $
    Inline equation example $f(x) = x^2 +4x -5$ test test. \\
    %   type 2 - enclose to \( \)
    Inline equation example \(f(x) = x^2 +4x -5\) test test. \\

    % - some equation in two mode have different style
    \begin{center}
        {\LARGE some equation in two mode have different notaion}\\
        \begin{align*}
            Display Style Math
            \sum_{n=1}^\infty \frac{1}{n^2} = \frac{\pi^2}{6}
        \end{align*}
        Inline or Text Style Math
        \(\sum_{n=1}^\infty \frac{1}{n^2} = \frac{\pi^2}{6}\)
    \end{center}
    % - Force display style math in inline mode: \( \displaystyle ...\)
    % - Force inline math in display mode: \[ \ textstyle ...\]


    \begin{center}
        {\LARGE Basic Math Notation\\ Arithmetic \\}
        \(1+1\)\\
        \(5-3\)\\
        \(6 \cdot 4\) \\
        \(6 \times 4\) \\
        \(27 \div 9 \) \\ %\newpage
        {\LARGE Fractions \\}
        \[\frac{numerator}{denominator}\]\\
        % Force Display style
        \[\dfrac{numerator}{denominator}\]\\
        % Force Inline
        \[\tfrac{numerator}{denominator}\]\\
    \end{center}


    \begin{center}
        {\LARGE Superscript and Subscript}\\[0.5cm]
        \emph{\Large{Use of Brackets for Grouping}}\\[1cm]
        with Brackets: \(e^{kx}\)\\[0.25cm]
        without Brackets: \(e^kx\)\\[1cm]
        \emph{\Large{Simutaneous Superscript and Subscript}}\\[0.5cm]
        a sub-1 squared: \(a_1^2\)\\[0.25cm]
        a squared sub-1: \(a^2_1\)\\[1cm]
        \emph{\Large{Combined Superscripts and Subscripts}}\\[0.5cm]
        Stacked: \(p_1^{a_1}\)\\[0.25cm]
        Offset: \({p_1}^{a_1}\)\newpage
    \end{center}


    \begin{center}
        {\LARGE Parentheses}\\[0.5cm]
        % type 1
        ( a )\\[0.25cm]
        % type 2
        \big( a \big)\\[0.25cm]
        % type 3
        \Big( a \Big)\\[0.25cm]
        % type 4
        \bigg( a \bigg)\\[0.25cm]
        % type 5
        \Bigg( a \Bigg)\\[0.25cm]
        % type 6 - \left and \right can  use only one side,just use blank symbol after \left or \right
        \[ \left( \frac{numerator}{denominator} \right) \]
    \end{center}

    \begin{center}
        {\LARGE Text in Math Mode}\\
        % way 1
        \[ n = ab \text{ where } a \text{ and } b \text{ are natural numbers} \]\\
        % way 2
        \[ n = ab \text{ where \( a \) and \( b \) are natural numbers } \]\\
    \end{center}

    \begin{center}
        {\LARGE Greek Letters}
        \begin{align*}
            \text{alpha: }& \alpha  A\\
            \text{beta: }& \beta  B\\
            \text{gamma: }& \gamma  \Gamma\\
            \text{delta: }& \delta  \Delta\\
            \text{epsilon: }& \epsilon  \varepsilon  E\\
            \text{zeta: }& \zeta  Z\\
            \text{eta: }& \eta  H\\
            \text{theta: }& \theta  \vartheta  \Theta\\
            \text{iota: }& \iota  I\\
            \text{kappa: }& \kappa  K\\
            \text{lambda: }& \lambda  \Lambda\\
            \text{mu: }& \mu  M\\
            \text{nu: }& \nu  N\\
            \text{xi: }& \xi  \Xi\\
            \text{omeicron: }& o  O\\
            \text{pi: }& \pi  \Pi\\
            \text{rho: }& \rho \varrho P\\
            \text{sigma: }& \sigma \Sigma\\
            \text{tau: }& \tau T\\
            \text{upsilon: }& \upsilon \Upsilon\\
            \text{phi: }& \phi \varphi \Phi\\
            \text{chi: }& \chi X\\
            \text{psi: }& \psi \Psi\\
            \text{omega: }& \omega \Omega\\
        \end{align*}
    \end{center}

    \begin{center}
        {\LARGE AMS Blackboard Font}
        \begin{align*}
            \text{Natual Number: }& \mathbb{N}\\
            \text{Integer: }& \mathbb{Z}\\
            \text{Rational Number: }& \mathbb{Q}\\
            \text{Real Number: }& \mathbb{R}\\
            \text{Complex Number: }& \mathbb{C}\\
        \end{align*}
    \end{center}
\end{document}