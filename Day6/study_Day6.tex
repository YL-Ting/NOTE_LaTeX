\documentclass{article}

\usepackage{geometry,amsmath,amssymb,graphicx}

\usepackage{tocloft} 
% This pkg can make more custom setting for contents 

\usepackage{hyperref}
% use this package automatically turns the table of contents into hyperlinks
% all use of \ref{} will also become hyperlinks
% use \ref*{} to make references without hyperlinks

\author{YL-TING}
\title{Learning LaTeX - Day6}
\date{\today}
\begin{document}
    \maketitle
    \newpage

    % "~"$text$ is called non-breaking Space,its can make text not isolated in newline

    % latex will automatically put a larger a space after every period treating it like the end of a sentence
    % if want a regular space,can put "\ " immediately after the period 

    {\Large Hyphenation}\\
    % can avoid the text overflow the document margin
    {\Large He screamed aaaaaaaaaaaaaaaaaaaaaaaaaaaaaaaaaaaaaaaaaaaaaaaaaaaaaaaa\-aaaaaaaaaaaa with 20 'a's.}\\[15pt]

    {\Large Horizontal and Vertical Space}\\
    % \hspace{length} can create a horizontal space of the indicated size
    % \vspace{length} can create a vertical space of the indicated size,usually use between two paragraph
    % hspace example
    Name : \underline{\hspace{6cm}}\\[15pt]

    % vspace example
    \begin{center}
        \LARGE My Paper's Title
    \end{center}
    \vspace{1cm}
    This is the start of my paper.There is a bigger gap between the title and text.\\[15pt]

    {\Large Phantom Characters}\\[15pt]
    % phantom characters are invisible text characters used to be align text
    % example
    \begin{tabular}{ll}
        Saturday at 10 AM & Breakfast \\
        \phantom{Saturday at} 1 PM & Lunch \\
        Sunday at 8 AM & Breakfast \\
        \phantom{Sunday at} 11 AM & Brunch 
    \end{tabular}\\[15pt]

    {\Large Small Caps Font}\\[15pt]
    % \textsc{} create the uppercase text and font size equal to regular lowercase text
    This is a mixture of REGULAR CAPS and \textsc{small caps} fonts\\[15pt]
    
    {\Large Boxes}\\[15pt]
    % use \fbox{} in text mode and \boxed{} in math mode
    A \fbox{theorem} is a proven mathmatical statement.\\
    $\boxed{3+\boxed{2 \cdot 5}}$\\[15pt]

    {\Large Line Fill and Dot Fill}\\[15pt]
    % \hrulefill creates a solid line
    % \dotfill creates a line od dots
    Fill by line \hrulefill\\
    Fill by dots \dotfill\\
    % can use as a sperate line of paragraph
    \noindent \hrulefill

    \newpage

    {\Large Creaing Lines}\\
    % \rule{width(horizontal length)}{thickness(vertical length)}
    \rule{\linewidth}{0.4pt}
    \begin{center}
        \rule{3cm}{0.5cm}
    \end{center}

    {\Large Import Images}\\
    % use graphicx package
    % support file type : JPEG,PNG,EPS,PDF
    % define photo size directly and keep the width-height ratio
    \includegraphics[width=6cm]{test_photo.png}\\
    \includegraphics[height=6cm]{test_photo.png}\\
    % ref to line width
    \includegraphics[width=0.5\linewidth]{test_photo.png}\\
    % ref to the original size of photo
    \includegraphics[scale=1.5]{test_photo.png}\\
    % the photo may distorted
    \includegraphics[width=0.5cm,height=1cm]{test_photo.png}\\
    
    \newpage

    {\Large Document Structure}
    % the section,subsection and subsubsection has own counter and will create the header
    \section[control section display in table of contents]{The First Section}
    \subsection{The First Subsection}
    \subsubsection{The First Subsubsection}
    \subsubsection{The Second Subsubsection}
    \subsection{The Second Subsubsection}
    \section{The Second Subsubsection}

    {\Large Table of Contents}\\
    % this command will foemat the document and figure out how long ecah section is
    % and then can go back and fill in the proper page number
    \tableofcontents
    % use [] can control what appears in the table of contents 


    \vspace{1cm}
    {\Large Labels and References}
    % when document getting long ,we need ref some contents of section
    % can use \label{reference_value} and \ref{reference_value}
    % the \label{} is put on the section want to ref ,based on whatever counter is being used for the object
    % the \ref{} will automatically insert the value associated with the label but only benerate the number of the reference
    \section{The Section} \label{sec:MySec}
    \subsection{The Subsection}\label{subsec:MySubsec}
    \subsubsection{The Subsubsection}\label{ss1}
    \subsubsection{The Subsubsection}\label{ss2}

    % there is a excellent place to use non-breaking space ~ 
    Section~\ref{sec:MySec}\\
    Subsection~\ref{subsec:MySubsec}\\
    Theorem\ref{ss1}\\

    \newpage

    {\Large Hyperlinks}\\[15pt]
    % should use hyperref package
    % use \hyperlink{label}{links text} for custom references
    \hyperlink{ss2}{click to jump ss2}\\[10pt]
    % to create internet hyperlinks : \href{URL}{text}
    \href{http://google.com}{google site}\\


    {\Large Using Multiple Files}
    % use \input{} and \include{} to link to other document
    % the \input{} just drops the contents of the file like copy and paste
    % the \include{} also execute some code in background and start in new page
    % have some limit : disable to have nested levels of include this
\end{document}