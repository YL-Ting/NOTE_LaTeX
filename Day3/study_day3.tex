\documentclass{article}

\usepackage{geometry}
\usepackage{amsmath,amssymb}
\usepackage{graphicx}
% bm present vector by boldface
\usepackage{bm}
% esvect present vector by rightarrow
\usepackage{esvect}

\author{YL-TING}
\title{Learning LaTeX - Day3}
\date{\today}
\begin{document}
    \maketitle
    \newpage
    {\Large Polynomial}
    \[ f(x) = a_n x^n + a_{n-1} x^{n-1} + \cdots + a_1 x + a_0 \]

    {\Large Exponentials}
    \[ f(x) = c_1 e^{r_1 x} +c_2 e^{r_2 x} \]

    {\Large Special Function}
    % special fuction
    \[ \sin(x) \]
    % without - this consider as a variable
    \[ sin(x) \]
    % custom function
    \[ \operatorname{fun}(x) \]
    % without - this consider as a variable
    \[ fun(x) \]

    {\Large Limit}\\
    \text{Display Style : } \[ \lim_{x \to \infty} \frac{x^2+1}{x^2-1}=1 \]
    \text{Inline Style : } 
    \begin{center}
        \( \lim_{x \to \infty} \frac{x^2+1}{x^2-1}=1 \)\\
        % force display style in inline mode
        \( \displaystyle \lim_{x \to \infty} \frac{x^2+1}{x^2-1}=1 \)
    \end{center}
    
    {\Large Summation}\\
    \text{Display Style : }
    \[ \sum_{n=1}^{\infty} \frac{1}{n} \]
    % when n have more than one require,use \substack{} to make multiline
    \[ \sum_{ \substack{n=0 \\ n \text{ odd}} }^\infty a_{n}x^n \]
    \text{Inline Style : }
    \begin{center}
        \( \sum_{n=1}^{\infty} \frac{1}{n} \)\\[10pt]
        \( \sum_{ \substack{n=0 \\ n \text{ odd}} }^\infty a_{n}x^n \)
    \end{center}

    % we define a new command \subalign to solve the multivariable align problem of substack
    \makeatletter
    \newcommand{\subalign}[1]{%
        \vcenter{%
            \Let@ \restore@math@cr \default@tag
            \baselineskip\fontdimen10 \scriptfont\tw@
            \advance\baselineskip\fontdimen12 \scriptfont\tw@
            \lineskip\thr@@\fontdimen8 \scriptfont\thr@@
            \lineskiplimit\lineskip
            \ialign{\hfil$\m@th\scriptstyle##$&$\m@th\scriptstyle{}##$\crcr #1\crcr}%
        }
    }
    \makeatother
    {\Large subalign vs substack}\\
    \text{subalign}
    \[ \sum_{\subalign{n&=0 \\ m&=0}}^\infty \]
    \text{substack}
    \[ \sum_{\substack{n=0 \\ m=0}}^\infty \]


    \newpage


    {\Large Integral } \\[5pt]
    {\normalsize Single Integral : } \( \int \)\\
    {\normalsize Double Integral : } \( \iint \)\\
    {\normalsize Triple Integral : } \( \iiint  \)\\
    {\normalsize Upper/Lower Limit Location : }\\
    % Inline mode
    \[ \textstyle \int_0^\infty \]
    \[ \textstyle \int \limits_0^\infty \]
    % Display mode
    \[ \displaystyle \int_0^\infty \]
    \[ \displaystyle \int \limits_0^\infty \]

    {\Large Spacing command}
    % type1 space
    \[ \iiiint f(x,y,z) dx dy dz \]
    % type2 \,
    \[ \iiiint f(x,y,z) \, dx \, dy \, dz \]
    % type3 \:
    \[ \iiiint f(x,y,z) \: dx \: dy \: dz \]
    % type4 \;
    \[ \iiiint f(x,y,z) \; dx \; dy \; dz \]
    % type5 \quad
    \[ \iiiint f(x,y,z) \quad dx \quad dy \quad dz \]

    % Some time in order to distinguish the operator and variable
    % we use \mathrm to modify derivative operator
    \[ \int f(x) \, \mathrm{d}x \]
    \[ \int f(x) \, dx \]

    % integral equation with limits
    \[ \displaystyle \int_a^b f(x) \, dx = F(x) \bigg\vert_a^b \]
    \[ \textstyle \int_a^b f(x) \, dx = F(x) \bigg\vert_a^b \]


    \newpage


    {\Large Derivative}\\
    \[ \frac{ \mathrm{d}f }{ \mathrm{d}x } \]
    \[ \frac{\partial f}{\partial x} \]
    {\normalsize Prime (Lagrangian) Notation}
    \[ f'(x) \]
    \[ f''(x) \]
    \[ f'''(x) \]
    \[ f^{(n)}(x) \]
    {\normalsize Dot (Newtonian) Notation}
    \[ \dot{x}(t) \]
    \[ \ddot{x}(t) \]
    \[ \dddot{x}(t) \]
    \[ \ddddot{x}(t) \]

    {\Large Vectors}
    % the value of vector should be enclosed by \langle and \rangle
    % use bm package
    \[ \bm{r}(t) = \langle x(t),y(t),z(t) \rangle \]
    % use esvect package
    \[ \vv{v_1}(t) = \langle x(t),y(t),z(t) \rangle \]
    \[ \vv*{v_1}(t) = \langle x(t),y(t),z(t) \rangle \]

    % example
    \[ \vv{\nabla} \times \vv{E} = - \frac{\partial \vv{B}}{\partial t} \]
    \[ \oint \vv{E} \cdot d\vv{s} = \dfrac{d\Phi_B}{dt} \]


\end{document}