\documentclass{article}
% can use option to define paper size
% avaliable option : a4paper, a5paper, b5paper, letterpaper, legalpaper, executivepaper
% a4 : 297mm*210mm or 11.69in*8.27in
% letterpaper : 279.4mm*215.9mm or 11in*8.5in


\usepackage{graphicx,amssymb,amsmath}

\usepackage{geometry}
% use geometry can define more kind of papersize
% avaliable option : [a-c][1-6]paper, b[0-6]j, ansi[a-e]paper, letterpaper, legalpaper, executivepaper, screen
% can also use paperwidth and paperheight in option to define custom paper size
% set margin of document : can use margin=value in option to define all margin
% or use top,bottom,left,right for each,sepreate by camma(,)
% if we hope the document is landscape layout can use landscape in option(for entire document)

\usepackage{pdflscape}
% this package can let one or more page use landscape layout by
% \begin{landscape} ... \end{landscape}

\usepackage{fancyhdr}
% can make layout fancy
\usepackage{lastpage}
% can display the last page number of document
\usepackage{enumitem}
% can order the number icon when use enumerate


\usepackage{amsthm} % can typesetting and auto-numbered theorem
% have 3 default theorem styles : plain, definition and remark , use \theoremstyle{} to setup
% \newtheorem{environmentname}{class(ex:Theorem,formula...)}
\theoremstyle{plain}
\newtheorem{thm}{Theorem}
% each \newtheorem will have independent counter
\theoremstyle{definition}
\newtheorem{lem}{Lemma}
% if want share counter can add [environmentname] after {environmentname}
\theoremstyle{remark}
\newtheorem{prop}[thm]{Proposition}
% if don't need counter ,use \newtheorem*{}{}
\newtheorem*{for}{Formula}




% below command can use in preamble and document part
% User-Define Commands
% \newcommand{\commandname}{definition} can define a new command for user
% if the commandname is already exist,use \renewcommand{\commandname}{definition} to define

% User-Define Function
% can use \newcommand{\myoperator}{\operatorname{myoperator}} or \DeclareMathOperator{\myoperator}{myoperator}
% Notice : if cammand name have * will have different text layout(above and below)
% example
\newcommand{\FunNS}{\operatorname{FunNS}} % equal to \DeclareMathOperator{\FunNS}{FunNS}

\newcommand{\FunS}{\operatorname*{FunS}} % equal to \DeclareMathOperator*{\FunS}{FunS}


% User-Define Environment
% \newenvironment{name}{begin_code}{end_code}
% example
\newenvironment{mythm}{\noindent \textbf{Theorem.} \begin{itshape}}{\end{itshape}}

\author{YL-TING}
\title{Learning LaTeX - Day5}
\date{\today}
\begin{document}
    \maketitle
    \newpage

    \begin{landscape}
        {\Large this is landscape layout}
    \end{landscape}

    \newpage
    use fancyhdr package 
    \pagestyle{fancy}
    \fancyhead[L]{Top Left}
    \fancyhead[C]{Top Center}
    \fancyhead[R]{Top Right}

    \fancyfoot[L]{Bottom Left}
    \fancyfoot[C]{Bottom Center}
    \fancyfoot[R]{Bottom Right}

    % \thepage can print current page number
    % if want display the lastpage,use lastpage package
    % the "\ " can let latex compiler know there has a space
    \fancyfoot[C]{Page \thepage \ of \pageref{LastPage}}

    % difference between \pagebreak and \newpage
    % \pagebreak will stretch out the vertical spacing so that text fill the page("Soft" page break,can keep the flow and continuity of the text)
    % \newpage will push all the space to the bottom("Hard" page break,create hard stop)
    \newpage


    % Notice : if cammand name have * will have different text layout(above and below)
    % example
    \[ \FunNS_a^b \]
    \[ \FunS_a^b \]

    % Basic LaTeX Environments
    %   document : Sets up the basic document structure
    %   center : Centers the text on the page
    %   flushright : Right-justifies the text on the page
    %   flushleft : Left-justifies the text on the page
    %   tabular : A text table environment
    %   array : A math table environment
    %   align : A numbered math environment with alignment
    %   align* : An unnumbered math environment with alignment
    %   itemize : create unnumbered list
    %   enumerate : create numbered list

    % \[ <--> \begin{displaymath}   \] <--> \end{displaymath}
    % \( <--> \begin{math}   \) <--> \end{math}

    % can change item icon by []
    \begin{itemize}
        \item First Item
        \item[$\rightarrow$] Second 
        \begin{itemize}
            \item[] First Sub-item
            \item Second Sub-item
        \end{itemize}
    \end{itemize}

    \begin{enumerate}
        \item[] First Item
        \item Second 
        \begin{enumerate}
            \item First Sub-item
            \item[$\leftarrow$] Second Sub-item
        \end{enumerate}
    \end{enumerate}

    % use enumitem package
    \begin{enumerate}[label=\arabic*)]
        \item First item
        \item Second item
    \end{enumerate}
    \begin{enumerate}[label=\Roman*]
        \item First item
        \item Second item
    \end{enumerate}
    \begin{enumerate}[label=\roman*]
        \item First item
        \item Second item
    \end{enumerate}
    \begin{enumerate}[label=(\alph*)]
        \item First item
        \item Second item
    \end{enumerate}

    % use amsthm package
    \begin{thm}
        First Theorem Statement
    \end{thm}
    \begin{thm}
        Second Theorem Statement
    \end{thm}

    \begin{lem}
        First Lemma Statement
    \end{lem}
    \begin{prop}
        First Proposition Statement
    \end{prop}
    \begin{for}[optional description]
        First Foumula
    \end{for}

    % proof statement doesn't need define in preamble
    % it will put a square box in the end of proof flush against the right
    % if want to change symbol, should have to redefine the QED symbol command,it can fit at the end of the current line
    \begin{proof}
        This is proof statement
    \end{proof}
    \renewcommand{\qedsymbol}{Q.E.D.}
    \begin{proof}
        This is proof with QED
    \end{proof}

    % User-Define Envionment
    \begin{mythm}
        theorem statement
    \end{mythm}
\end{document}