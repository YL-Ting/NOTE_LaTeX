\documentclass{beamer}
% beamer is a document class for making presentation slides(like PPT)
% beamer automatically loads the amsthm environment so can use \newtheorem
\newtheorem{thm}{Theorem}


\beamertemplatenavigationsymbolsempty
% the symbols in bottom right called navigation symbols
% this command can make navigation symbols disappear

\usetheme{Boadilla} % Themes are location names ,the first letter of themes name should use uppercase
\usecolortheme{whale} % Color themes are animal names
% below is the website that can view all theme and colortheme of latex
% https://hartwork.org/beamer-theme-matrix/
% https://deic-web.uab.cat/~iblanes/beamer_gallery/index.html

\usepackage{geometry,graphicx,amsmath,amssymb}



\author{YL-TING}
\title{Learning LaTeX - Day7}
\date{\today}
\begin{document}
    \section{sec1}
    % In beamer ,use frame environment to present one or more pages of slides
    \begin{frame}
        \frametitle{Frame Title}
        \maketitle
    \end{frame}

    \begin{frame}
        \frametitle{Some Math Stuff}
        Here is some text followed by an equation.
        \[ax^2+bx+c=0\]
        \begin{itemize}
            \item First 
            \item Second
        \end{itemize}
    \end{frame}

    \begin{frame}
        \frametitle{Creating Columns}
        % if need to make multiple columns of text can use columns environment
        % each column will vertically center
        This is the text above the columns
        \vspace{5mm}

        \begin{columns}
            
            \column{0.45\linewidth}
            This is the First column

            \column{0.45\linewidth}
            This is the Second column

        \end{columns}

        \vspace{5mm}
        This is the text below the columns
    \end{frame}

    \begin{frame}
        \frametitle{Pausing Text}
        Here is some more text
        \pause
        % this command will display everything the frame up to the command of one slide
        % and then create a second slide immediately afterwards that has full contents(like PPT's animation)
        Here is some more text
    \end{frame}

    % Specification Overlays
    % Define the slides in the frame where the feature applies(like PPT's animation)
    % for example:
    %   <2> : slides 2 only
    %   <1-4> : Slides 1 through 4,inclusive
    %   <3-> : Slides 3 through the end
    %   <1,3-4,6-> : Slides 1,3,4,6 and throu the end
    \begin{frame}
        \frametitle{Specification Overlays and Itemize}
        \begin{itemize}
            \item<1-> This appears on all frames.
            \item<2-3> This appears on frames 2 and 3
            \item<2> This appears on the second frame only.
            \item<2,4> This appears on frames 2 and 4.
        \end{itemize}
    \end{frame}
    \subsection{ssec1}
    \begin{frame}
        \frametitle{Specification Overlays and Text Styles}
        This is \textbf<2>{bolded} text\\
        This is \textit<2>{italiticized} text\\
        This is \textcolor<2>{blue}{colorful} text\\
        This is \uncover<2>{uncover} text\\
        This is \only<2>{appearing} text with a double space\\
        This is\only<2>{ appearing} text without a double space\\
    \end{frame}

    \begin{frame}
        \frametitle{Blocks}

        \begin{block}{Block Title}
            This is a standard block
        \end{block}

        \begin{thm}[Theorem Name]
            This is a theorem block
        \end{thm}

        \begin{proof}
            This is a proof block
        \end{proof}
    \end{frame}
    \subsection{ssec2}
    \begin{frame}
        % the option will highlight the current section/subsection and fade everything else 
        \tableofcontents[currentsection,currentsubsection]
    \end{frame}
\end{document}