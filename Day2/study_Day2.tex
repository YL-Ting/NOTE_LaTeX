\documentclass{article}

\usepackage{geometry}
\usepackage{amsmath,amssymb}
\usepackage{graphicx}
\usepackage{hhline} %to make table beauty
% more useful table package : booktabs,colortbl,tabularx,longtable
\usepackage{mathdots} %can use more different dots
\usepackage{multirow} % can merge rows

\author{YL-TING}
\title{Learning LaTeX - Day2}
\date{\today}

\begin{document}
    \maketitle
    \newpage
    \begin{center}
        {\LARGE Table and Array}\\[15pt]
        % in LaTex ,use table or tabular environment to create table
        % use array environment to create array

        {\LARGE Table}\\[5pt]
        \begin{tabular}{lcr} 
            % lcr mean this table has 3 columns
            % left justified for 1st col , center justified for 2nd col , right justified for 3rd col
            % use & to align the content
            % in Latex ,numbers of space will consider to one space
            text & text & text\\
            l    & c    & r
        \end{tabular}\\[15pt]


        \begin{tabular}{|lc||r|}
            % we can use | to make vertical bar between two columns
            % use \hline to make horizontal bar
            \hline
            text \vline & text & text\\
            l    \vline & c    & r\\
            \hline \hline
        \end{tabular}\\[15pt]

        % after use hhline package
        \begin{tabular}{|cc|c|}
            11 & 12 & 13 \\
            \hhline{|-~|=|}
            21 & 22 & 23
            
            %   option of hhline:
            % = : a double hline the width of a column
            % - : a single hline the width of a column
            % ~ : a column with no hline
            % | : a vline which 'cuts' through a double(or single) hline
            % : : a vline which is broken by a double hline
            % # : a double segment between two vlines
            % t : the top half of a double hline segment
            % b : the bottom half of a double hline segment
        \end{tabular}\\[15pt]

        % Merging Columns and Rows
        \begin{tabular}{|c|c|c|}
            \hline
            very long text & very long text & very long text\\
            \hline
            \multicolumn{3}{|c|}{Center Merged}\\
            \hline
            Text & \multicolumn{2}{|l|}{Left Aligned Merged}\\
            \hline
        \end{tabular}\\[15pt]

        % use multirow package to able \multirow
        \begin{tabular}{|c|c|}
            \hline
            % \multirow[valign]{num_rows}{width}{contents}
            % valign can be t(top),c(center),b(bottom)
            % width general use *(auto) ,or length (ex:1cm)
            \multirow{2}{*}{Row Merge} & Text \\
            & Text \\
            \hline
        \end{tabular}\\[10pt]

        % example for merge
        \begin{tabular}{|c|c||c|c|}
            \hhline{~~|--}
            \multicolumn{2}{c|}{} & \multicolumn{2}{c|}{Game}\\
            \hline
            Team & Player & 1 & 2\\
            \hline
            \multirow{2}{*}{A} & A1 & 2 & 3\\
            \hhline{|~|---|}
            & A2 & 0 & 2\\
            \hhline{|=|=#=|=|}
            \multirow{2}{*}{B} & B1 & 1 & 0\\
            \hhline{|~|---|}
            & B2 & 3 & 1\\
            \hline
        \end{tabular}\\[15pt]

        {\LARGE Array}\\[5pt]
        {\normalsize Array MUST be in math mode}\\[5pt]
        \begin{align*}
            \begin{array}{c|cc}
                a_{11} & a_{12} & a_{13}\\
                \hline
                a_{21} & a_{22} & a_{23}\\
            \end{array}
        \end{align*}\\[5pt]
        
        % Matrix 
        % type1: can also use \left( and \right) as notation
        \begin{align*}
            \left[ \begin{array}{ccc}
                a_{11} & a_{12} & a_{13}\\
                a_{21} & a_{22} & a_{23}\\
            \end{array} \right]
        \end{align*}\\[5pt]
        % type2: pmatrix
        \begin{align*}
            \begin{pmatrix}
                a_{11} & a_{12} & a_{13}\\
                a_{21} & a_{22} & a_{23}\\
            \end{pmatrix}
        \end{align*}\\[5pt]
        % type3: bmatrix
        \begin{align*}
            \begin{bmatrix}
                a_{11} & a_{12} & a_{13} & \cdots\\
                a_{21} & a_{22} & a_{23} & \cdots\\
                \vdots & \vdots & \vdots & \ddots\\
                % only for use mathdots package
                \iddots & \iddots & \iddots & \iddots
            \end{bmatrix}
        \end{align*}\\[5pt]
    \end{center}
\end{document}